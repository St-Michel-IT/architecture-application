\documentclass{beamer}

\include{./cours-style.sty}

% Title
\title{Architectures d’Applications - Bachelor CSI}
\author{Christophe Brun}
\institute{Campus Saint-Michel IT}
\date{03 avril 2024}
\beamertemplatenavigationsymbolsempty

% Graphix with arrows in between
\newcommand*{\vcenterimage}[1]{\vcenter{\hbox{\includegraphics[width=5cm]{#1}}}}
\newcommand*{\vcenterarrow}{\vcenter{\hbox{$\Longrightarrow$}}}

\titlegraphic{
    \bigbreak
    \includegraphics[width=2cm]{image/logo-papit}
    \includegraphics[width=2cm]{image/logo-campus-saint-michel-it}
}
\begin{document}

    \begin{frame}
        \transdissolve
        \titlepage
    \end{frame}

    \begin{frame}{Table des matières}
        \tableofcontents
    \end{frame}


    \section{Programme du module}\label{sec:programme-du-module}
    \begin{frame}
        \frametitle{Architectures d’Applications}
        \framesubtitle{Compétence acquise aucours des 3 jours du module}
        \transdissolve
        Compétences~:
        \begin{itemize}
            \item \textquote{Concevoir une architecture d’applications.}~???
        \end{itemize}
        \pause
        \bigbreak
        Reformulation~:
        \begin{itemize}
            \item Concevoir une application architecturée.
        \end{itemize}
        \centering
        \includegraphics[width=5cm]{image/engraving-of-a-monk-drawing-a-cathedral}
    \end{frame}

    \begin{frame}
        \frametitle{Architectures d’Applications}
        \framesubtitle{Le programme officiel des 3 jours du module}
        \transdissolve
        \fontsize{8pt}{8pt}\selectfont
        \begin{enumerate}
            \item Les différentes architectures d’une application
            \item L’architecture REST
            \begin{itemize}
                \fontsize{8pt}{8pt}\selectfont
                \item Architectures Orientées Services
                \begin{itemize}
                    \item Besoins de la SOA
                    \item Notion de service
                    \item Introduction aux Architectures Orientées Services
                \end{itemize}
                \item Vers les Architectures Orientées Services
                \begin{itemize}
                    \item Les architectures Client-Serveur
                    \item Les architectures Web
                \end{itemize}
                \item Les Web Services
                \begin{itemize}
                    \item Appel de procédure
                    \item World Wide Web
                    \item Formats d’échange textuels
                    \item Vers la notion de Web Service
                \end{itemize}
                \item Web Service de type SOAP et REST
                \item Guidelines API-REST
                \begin{itemize}
                    \item Gestion des actions et des URLs
                    \item Recherche, Tri, Filtre et Pagination
                    \item Gestion des erreurs
                \end{itemize}
            \end{itemize}
        \end{enumerate}
    \end{frame}

    \begin{frame}
        \transdissolve
        \frametitle{Evaluation}
        \begin{itemize}
            \item Bons points tout au long du module.
            \item 25 \% x 3 sur chaque séance de travaux dirigés évalués en fin de séance.
            2 architectures agnostiques à une application et une orientée web REST ou SOAP~.
            \item 25 \% sur une évaluation écrite finale
        \end{itemize}
    \end{frame}

    \begin{frame}
        \transdissolve
        \frametitle{Intervenant sur le module Architecture d'Aplication}
        \framesubtitle{Christophe Brun, conseil en développement informatique}

        \begin{columns}
            \column{0.7\textwidth}
            \begin{itemize}
                \item 1\textsuperscript{ere} année d'intervenant à Saint-Michel \emoji{star-struck}.

                \item 7 ans de conseil en développement au sein d'SSII~.

                \item 7 ans de conseil en développement à mon compte \href{https://papit.fr}{PapIT}.

                \item Passionné~!
                \bigbreak
                \begin{columns}
                    \column{0.5\textwidth}
                    \centering
                    \includegraphics[width=3cm]{image/logo-uppa}
                    \column{0.5\textwidth}
                    \centering
                    \includegraphics[width=3cm]{image/logo-universite-bordeaux}
                \end{columns}
            \end{itemize}
            \column{0.3\textwidth}
            \centering
            \includegraphics[width=5cm]{image/trombine-christophe}
        \end{columns}
    \end{frame}


    \section{Généralités}\label{sec:generalites}

    \begin{frame}
        \transdissolve
        \frametitle{Pourquoi le design ou Architecture logicielle~?}
        \begin{columns}
            \column{0.5\textwidth}
            \textquote{The goal of software architecture is to minimize the human resources required to build and maintain the required system.}\footnotemark
            \centering
            \column{0.5\textwidth}
            \centering
            \begin{tabular}{cc}
                \includegraphics[width=6.0cm]{image/da-vinci-plans} \\
                L. de Vinci par C. G. Gerli en 1789.                \\
            \end{tabular}
        \end{columns}
        \footnotetext{\label{cleanarchitecture}Clean Architecture, Robert C. Martin}
    \end{frame}

    \begin{frame}
        \transdissolve
        \frametitle{Eloge de la simplicité}
        Léonard de Vinci, 1452-1519~: \textquote{La simplicité est la sophistication suprême.}\footnote{La Pause Philo, \url{https://lapausephilo.fr/2015/09/15/simplicite-sophistication-supreme-leonard-de-vinci/}}
        \bigbreak
        Albert Einstein, 1879-1955~: \textquote{Si vous ne pouvez expliquer quelque chose simplement, c’est que vous ne l’avez pas bien compris.}
        \bigbreak
        Trouver et implémenter une Architecture logicielle doit être simple.
        C'est compliqué il y a un souci.
        \bigbreak La modularité est en autre permise par la dissection d'un système complexe en sous-systèmes plus simples.
        Ces systèmes plus simples n'ont qu'une seule responsabilité (Comme en programmation fonctionnelle~?).
    \end{frame}

    \begin{frame}
        \transdissolve
        \frametitle{Définition}
        \framesubtitle{Architecture~?, Design~?, les deux~?}
        Encore une fois, pour faire simple, nous allons présumer que l'architecture et le design sont la même chose dans le monde du logiciel.\cref{cleanarchitecture}
        Le \textquote{Design patterns}\footnote{Design patterns, \url{https://refactoring.guru/design-patterns}} est un classique en développement logiciel.
        \bigbreak
        \textquote{The onlye way to go fast is to go well.}\cref{cleanarchitecture}
        \bigbreak
        \textquote{The blueprint of the system.}\footnote{\label{fundsofsoftarch}Fundamentals of Software Architecture, Mark Richards et Neal Ford}
        \bigbreak
        Le mot système revient souvent.
        C'est un terme vague et récursif, c'est-à-dire qu'un système est composé de système plus simples.
        L'architecture système est d'ailleurs un domaine en charge de l'architecture de l'entreprise, des produits et des logiciels.
    \end{frame}

    \begin{frame}
        \transdissolve
        \frametitle{Définition}
        \framesubtitle{Le couplage}
        C'est un des termes les plus importants en Architecture logicielle.
        Il faut le comprendre l'éviter tant que possible.
        \bigbreak
        \textquote{It is a result of putting avariable, constant, or function in a temporary convinient, thought inappropriate, location. This is lazy and careless}\footnote{\label{cleancode}Robert C. Martin, Clean Code}
    \end{frame}


    \section{Les briques}\label{sec:les-briques}

    \begin{frame}
        \transdissolve
        \frametitle{Les briques}
        \framesubtitle{Les briques en architecture}
        \begin{columns}
            \column{0.6\textwidth}
            Probablement hérité du jargon de l'architecture des bâtiments, le mot brique est aussi souvent utilisé en Architecture logicielle.
            \bigbreak
            Beaucoup d'industries utilisent des briques pour construire des systèmes plus complexes.
            Parfois appelée \textquote{conception modulaire}, c'est par exemple, dans l'industrie automobile, réutiliser une même pièce sur plusieurs modèles de voitures.
            \bigbreak
            Quelles sont les briques en Architecture logicielle~?
            \column{0.4\textwidth}
            \centering
            \includegraphics[width=4cm]{image/engraving-of-a-craftman-cutting-a-stone}
        \end{columns}
        \bigbreak
        \flushleft
        1 des 4 règles pour la direction de l'esprit de Descartes~: \textquote{Diviser chacune des difficultés que j'examinerais, en autant de parcelles qu'il se pourrait et qu'il serait requis pour les mieux résoudre.}.
    \end{frame}

    \begin{frame}
        \transdissolve
        \frametitle{Les briques}
        \framesubtitle{3 paradigmes de programmation représentant par des briques\cref{cleanarchitecture}}
        \begin{itemize}
            \item Programmation structurée (sous-ensemble de la programmation impérative), où la brique est le module de code, i.e., de programmes et sous-programmes.
            Il contient les instructions de control de flux (if, else, for, while, etc.).
            \item Programmation orientée objet, où la brique est la classe.
            \item Programmation fonctionnelle, où la brique est la fonction.
        \end{itemize}
        \bigbreak
        La plupart des langages de programmation modernes supportent ces 3 paradigmes, JS, Python, Java, C/C++.
    \end{frame}

    \begin{frame}
        \transdissolve
        \frametitle{Les briques}
        \framesubtitle{Exemple de programmation structurée}
        Utilisation de tout le control flow classique.
        \bigbreak
        \centering
        \begin{tabular}{cc}
            \includegraphics[width=10cm]{image/analyse-topdown-prog-struc}                                                                                                                                        \\
            Analyse top-down pour division des taches en sous-programmes\footnote{Programmation structurée, \url{https://perso.univ-lyon1.fr/marc.buffat/COURS/BOOK_INPROS_HTML/CHAP5/COURS_ALGORITHMIQUE.html}}.                     \\
        \end{tabular}
    \end{frame}

    \begin{frame}[fragile]
        \transdissolve
        \frametitle{Les briques}
        \framesubtitle{Exemple de programmation fonctionnelle\cref{cleanarchitecture}}
        Les briques sont de simples fonctions, des fonctions dîtes pures.
        Les données ne sont modifiées uniquement que par ces fonctions.
        \bigbreak
        Des exemples de langages fonctionnels~: Haskell, Lisp, Erlang, Scala, F\#.
        \bigbreak
        D'autres langages supportent la programmation fonctionnelle comme Python, JS, Java en autre ont des outils orientés programmation fonctionnelle.
        Souvent nommées fonctions dîtes lambda ou \textquote{arrow functions}, map, reduce, filter, etc.
        \begin{lstlisting}[language=python]
>>> is_even = lambda x: x % 2 == 0 # No return needed?
>>> list(map(is_even, [1, 2, 3, 4]))
[False, True, False, True]
>>> list(filter(is_even, [1, 2, 3, 4])) # Kept only if True
[2, 4]
>>> from functools import reduce
>>> from operator import mul
>>> reduce(mul, [1, 2, 3, 4]) # Function applied to all elements, 1 * 2 * 3 * 4
24
        \end{lstlisting}
    \end{frame}

    \begin{frame}[fragile]
        \transdissolve
        \frametitle{Les briques}
        \framesubtitle{Exemple de programmation fonctionnelle}
        3 modules dédiés à la programmation fonctionnelle dans la \textquote{standard library}\footnote{Functional Programming Modules, \url{https://docs.python.org/3/library/functional.html}}~:
        \begin{itemize}
            \item \lstinline{functools}, \textquote{Higher-order functions and operations on callable objects}.
            \item \lstinline{itertools}, \textquote{Functions creating iterators for efficient looping}.
            \item \lstinline{operator}, \textquote{Standard operators as functions}.
        \end{itemize}
        \bigbreak
        L'équivalent en JS, sans aucune librairie à importer~:
        \begin{lstlisting}[language=java]
> numbers = new Array(1, 2, 3, 4, 5)
[ 1, 2, 3, 4, 5 ]
> const isEven = x => x % 2 === 0;
undefined
> numbers.map(isEven);
[ false, true, false, true, false ]
> numbers.reduce(isEven);
true
> numbers.filter(isEven);
[ 2, 4 ]
> numbers.reduce((x, y) => y * x);
120
        \end{lstlisting}
    \end{frame}

    \begin{frame}[fragile]
        \transdissolve
        \frametitle{Les briques}
        \framesubtitle{Exemple de programmation orientée objet\cref{cleanarchitecture}}
        La brique de base en programmation orientée objet est la classe.
        \bigbreak
        En OOP on parle de polymorphisme, c'est la capacité d'un objet, i.e., une classe, à prendre plusieurs formes.
        Une évolution dans une classe fille n'implique pas de modification dans la classe mère.
        Une évolution dans une classe mère impacte les classes filles.
        Si ces classes sont dans des modules différents, il n'est même pas nécessaire de recompiler tous les modules.
        Pouvoir controller quelle classe dépend de quelle autre est un atout majeur de cette Architecture logicielle.
        \bigbreak
        L'encapsulation des données et des méthodes dans une classe permet un design simple de ce qui appartient à un objet, une instance de classe, et ce qui est accessible depuis l'extérieur.
        \bigbreak
        L'encapsulation et l'héritage qui permet le polymorphisme sont les 2 atouts principaux de la programmation orientée objet.
    \end{frame}

    \begin{frame}[fragile]
        \transdissolve
        \frametitle{Les briques}
        \framesubtitle{Exemple de programmation orientée objet}
        Polymorphisme par héritage d'une classe.
        \begin{lstlisting}[language=python]
>>> class Oiseau:
        def __init__(self, name):
            self.name = name

>>> class Pigeon(Oiseau):
        def __init__(self, name):
            Oiseau.__init__(self, name)
        def mange_un_filtre_cigarette(self):
            print("{} mange un filtre de cigarette!!!".format(self.name))

>>> class Cygne(Oiseau):
        def __init__(self, name):
            Oiseau.__init__(self, name)
        def plonge(self):
            print("{} plonge dans le lac!!!".format(self.name))

>>> goelan = Oiseau("Jonathan Livingston")
>>> cygne = Cygne("Juste Leblanc") # C'est un Oiseau aussi !
>>> pigeon = Pigeon("Casimir") # Même le pigeaon est un Oiseau !
>>> all(map(lambda x: isinstance(x, Oiseau), [cygne, pigeon, goelan]))
True
        \end{lstlisting}
    \end{frame}

    \begin{frame}[fragile]
        \transdissolve
        \frametitle{Les briques}
        \framesubtitle{Exemple de programmation orientée objet}
        Polymorphisme par héritage d'une classe abstraite.
        \begin{lstlisting}[language=python]
>>> from abc import ABC, abstractmethod

>>> class Oiseau(ABC): # ABC -> ABstract Class
        def __init__(self):
            pass
        @abstractmethod
        def vole(self): # Polymorphisme au niveau de cette méthode !
            raise NotImplementedError

>>> class Poule(Oiseau):
        def __init__(self):
            Oiseau.__init__(self)
        def vole(self): # Chacun vole à sa manière, son implémentation
            print("Je vole quelques mètres")

>>> cocote = Poule()

>>> cocote.vol()
Je vole quelques mètres
        \end{lstlisting}
    \end{frame}

    \begin{frame}[fragile]
        \transdissolve
        \frametitle{Les briques}
        \framesubtitle{Exemple de programmation orientée objet}
        Contrainte d'implémentation d'une méthode abstraite \lstinline{vole} .

        \begin{lstlisting}[language=python]
>>> class Dodo(Oiseau):
        def __init__(self):
            Oiseau.__init__(self)

>>> Aussie = Dodo()
---------------------------------------------------------------------------
TypeError                                 Traceback (most recent call last)
Cell In [10], line 1
----> 1 Aussie = Dodo()

TypeError: Can't instantiate abstract class Dodo with abstract method vole
        \end{lstlisting}
    \end{frame}

    \begin{frame}
        \transdissolve
        \frametitle{Les types de briques}
        \framesubtitle{Exercice de 5 minutes}
        Trouvez quel type de paradigme de programmation est utilisé dans les exemples suivants~:
        \begin{itemize}
            \item \url{https://github.com/facebook/Haxl/}
            \item \url{https://github.com/gurupratap-matharu/Bike-Rental-System/blob/master/main.py}
            \item \url{https://github.com/papit-fr/fsma/}
        \end{itemize}
        Tirage au sort d'un étudiant pour chaque exemple, il doit expliquer pourquoi il pense que c'est ce type de programmation.
    \end{frame}


    \section{TDD}

    \begin{frame}
        \transdissolve
        \frametitle{Le TDD}
        \framesubtitle{Test Driven Development}
        Vouloir aller plus vite sans faire de test semble être une mauvaise idée.

        Ce qui pratiquent le TDD vont plus vite dès le début d'un projet\cref{cleanarchitecture}~!

        Entre autre, les tests unitaires font réduire la taille des classes, functions et des méthodes ce qui améliore architecture.
        \bigbreak
        \centering
        \includegraphics[width=8cm]{image/tdd-vs-no-tdd}
    \end{frame}

    \begin{frame}
        \transdissolve
        \frametitle{Le TDD}
        \framesubtitle{Quel type(s) de test~?}
        Tous les types de tests possibles sont à écrire dès que possible avant même de commencer à coder.
        Les tests unitaires sont en général les plus simples à écrire et les plus rapides à exécuter.
        \bigbreak
        Les éléments centraux, les règles de gestions, les \textquote{business rules} doivent être testables sans les éléments périphériques\cref{cleanarchitecture}~:
        \bigbreak
        \centering
        \includegraphics[width=6cm]{image/the-clean-architecture}
    \end{frame}

    \begin{frame}
        \transdissolve
        \frametitle{Le TDD}
        \framesubtitle{Quel types de test~?}
        Edsger Dijkstra~: \textquote{Testing shows the presence, not the absence of bugs.}
        \bigbreak
        \begin{columns}
            \column{0.5\textwidth}
            Autrement dit, l'univers des tests est infini, on n'a jamais la certitude de suffisamment tester.
            \bigbreak
            Même le coverage est une mauvaise mesure.
            \column{0.5\textwidth}
            \centering
            \includegraphics[width=5cm]{image/monk-looking-the-deepness-of-the-stars-in-the-night-sky}
        \end{columns}
    \end{frame}

    \begin{frame}
        \transdissolve
        \frametitle{Le TDD}
        \framesubtitle{Exercices}
        De nombreux exercices de TDD sont disponibles sur le web comme par sur GitHub, \url{https://github.com/gabbloquet/entrainement-au-tdd}.
        \bigbreak
        En 30 minutes, sans assistant du type ChatGPT, réaliser un des plus classique d'entre eux, le \textquote{FizzBuzz}, \url{https://github.com/gabbloquet/entrainement-au-tdd/blob/master/src/main/java/io/github/gabbloquet/tddtraining/FizzBuzz/FizzBuzz.java}~.
        Respectez l'esprit du TDD qui veut que l'on écrive les tests avant le code~:
        \begin{enumerate}
            \item Écrire un test par requirement.
            \item Écrire l'algorithme.
            \item Refactoriser.
            \item Revenir à l'étape 2 jusqu'à ce que tous les test passent.
        \end{enumerate}

    \end{frame}


    \section{Les mesures}\label{sec:les-mesures}

    \begin{frame}
        \transdissolve
        \frametitle{Les mesures du couplage}
        \framesubtitle{L'importance des données}
        L'architecture c'est le monde du compromis, des possibilités infinies.
        \bigbreak
        L'intuition, l'expérience, les connaissances et le bon sens, nous guiderons.

        Mais il est possible de mesurer l'architecture et donc d'avoir une approche \textquote{Data Driven} .
        \bigbreak
        Nous verrons dans cette section les différentes des métriques classiques, mais pas toutes, car elles sont nombreuses.

        On parle parfois des CK (Chidamber \& Kemerer) metrics qui peuvent ressembler, etc.
    \end{frame}

    \begin{frame}
        \transdissolve
        \frametitle{Les mesures du couplage}
        \framesubtitle{L'\textquote{abstractness} $A$\cref{fundsofsoftarch}}
        L'\textquote{abstractness}  une mesure de l'abstraction par rapport aux implémentations concrètes.
        Elle fut introduite par Robert C. Martin.
        \begin{equation}
            A = \frac{\sum Ca}{\sum Cc}
        \end{equation}
        Dans l'équation 1~:
        \begin{itemize}
            \item $A$ est \textquote{abstractness}
            \item $Ca$ est le nombre de classes abstraites
            \item $Cc$ est le nombre de classes concrètes.
        \end{itemize}
        \bigbreak
        Les valeurs de $A$ sont entre 0 et 1.
        Les extrêmes proches de 0 ou 1 sont à éviter, elles représentent des architectures trop concrètes ou trop abstraites.
    \end{frame}

    \begin{frame}
        \transdissolve
        \frametitle{Les mesures du couplage}
        \framesubtitle{Couplage afférent}
        Le couplage afférent est le nombre total de classes qui dépendent de la classe A~:
        \bigbreak
        \centering
        \includegraphics[width=5cm]{image/afferent-coupling.drawio}
        \bigbreak
        \flushleft
        Ici $Ca$ est 3.

        Un $Ca$ élevé impacte principalement la portabilité.
        Car ce code viendra forcément avec beaucoup de code en dépendance.
    \end{frame}

    \begin{frame}
        \transdissolve
        \frametitle{Les mesures du couplage}
        \framesubtitle{Couplage efférent $Ce$}
        Le couplage efférent est le nombre total de classes dont dépend de la classe A~:
        \bigbreak
        \centering
        \includegraphics[width=5cm]{image/efferent-coupling.drawio}
        \bigbreak
        \flushleft
        Ici $Ce$ est 3.

        Plus le $Ce$ est élevé, plus le code est difficile à reuse et à maintenir.
    \end{frame}

    \begin{frame}
        \transdissolve
        \frametitle{Les mesures du couplage}
        \framesubtitle{L'\textquote{instability} $I$\cref{fundsofsoftarch}}
        L'\textquote{instability} est une autre mesure qui en découle.
        Elle détermine la volatilité du code, l'effort à fournir pour modifier une partie du code sans devoir en modifier une autre.
        \begin{equation}
            I = \frac{Ce}{Ce + Ca}
        \end{equation}
        Dans l'équation 2~:
        \begin{itemize}
            \item $I$ est \textquote{instability}
            \item $Ca$ couplage afférent, objet ou package en entrée (\textquote{a} en premier d'où \textquote{entrée}).
            \item $Ce$ couplage efférent, objet ou package en sortie (\textquote{e} comme \textquote{exit}).
        \end{itemize}
        \bigbreak
        Quand elle est trop élevée, qu'elle tend vers 1, le code casse vite à cause d'un fort couplage.
        Elle a donc un impact négatif sur le reuse, les correctifs, la maintenance et la portabilité.
    \end{frame}

    \begin{frame}
        \transdissolve
        \frametitle{Les mesures du couplage}
        \framesubtitle{La\textquote{Distance from main sequence} $D$\cref{fundsofsoftarch}\textsuperscript{,}\footnote{A Study on Robert C.Martin’s Metrics for Packet Categorization
        Using Fuzzy Logic, Gurpreet Kaur and Deepak Sharma\url{https://gvpress.com/journals/IJHIT/vol8_no12/15.pdf}}}
        Une cinquième métrique introduite par Robert C. Martin est la \textquote{Distance from main sequence}.
        \bigbreak
        Elle définit une relation entre $A$ et $I$ comme suit~:
        \begin{columns}
            \column{0.5\textwidth}
            \begin{equation}
                D = |A + I - 1|
            \end{equation}
            \column{0.5\textwidth}
            \centering
            \includegraphics[width=5cm]{image/distance-from-main-sequence}
        \end{columns}
        \bigbreak
        \flushleft
        $D$ doit s'approcher de 0 pour une architecture souhaitable.
        Si $D$ est élevé, on s'éloigne de compromis acceptable.
    \end{frame}

    \begin{frame}
        \transdissolve
        \frametitle{Les mesures du couplage}
        \framesubtitle{Exercice de 30 minutes}
        Calculer les valeurs de $Ca$, $Ce$, $A$, $I$ et $D$ pour les chaque classe du source \url{https://github.com/St-Michel-IT/architecture-application/blob/main/coupling.py}.
        \bigbreak
        Restituez le résultat dans un tableau Excel et un graphique avec les $D$ et la droite \textquote{main sequence}.
    \end{frame}

    \begin{frame}
        \transdissolve
        \frametitle{Les mesures du couplage}
        \framesubtitle{Exercice de 30 minutes}
        Résultat de l'exercice~:
        \bigbreak
        \centering
        \includegraphics[width=8cm]{image/exercice-metrics-1}
    \end{frame}

    \begin{frame}[fragile]
        \transdissolve
        \frametitle{Les mesures du couplage}
        \framesubtitle{Les outils}
        Certains outils comme \lstinline{JCAT}\footnote{Tool for Measuring Coupling in Object-Oriented Java Software, \url{https://shorturl.at/npDR2}} en Java ou \lstinline{module_coupling_metrics}\footnote{\url{https://github.com/Oaz/module_coupling_metrics}} en Python permettent de calculer ces métriques.
        % shell listing with the command~:
        \begin{lstlisting}[language=bash]
$ module_coupling_metrics data_management.py
        \end{lstlisting}
        \begin{columns}
            \column{0.5\textwidth}
            \centering
            \includegraphics[width=5cm]{image/data-management-metrics-table}
            \column{0.5\textwidth}
            \centering
            \includegraphics[width=5cm]{image/data-management-disance-chart}
        \end{columns}

        Attention à bien comprendre ces métriques et comment les interpréter.
        \lstinline{module_coupling_metrics} ne calcule le couplage qu'entre modules.
        \bigbreak
    \end{frame}

    \begin{frame}
        \transdissolve
        \frametitle{Les mesures de la complexité cyclomatique}
        \framesubtitle{La théorie}
        Établie en 1976, elle est devenue le standard de mesure de complexity du code.
        \bigbreak
        Elle se calcule pour une méthode ou fonction avec l'équation suivante~:
        \begin{equation}
            CC = E - N + 2
        \end{equation}
        Ou~:
        \begin{itemize}
            \item $CC$ est la complexité cyclomatique.
            \item $E$ comme \textquote{edge} , est le nombre d'arêtes (les \lstinline{return} , \lstinline{yield}, \lstinline{exit}) du graphe de contrôle de flux.
            \item $N$ comme \textquote{node} , est le nombre de nœuds, les \lstinline{if} , \lstinline{while} , du graphe de contrôle de flux.
        \end{itemize}
        \bigbreak
        Les valeurs au-dessous de 5 sont bonnes, au-dessus de 10 il y a danger\cref{fundsofsoftarch} .
        \bigbreak
        Une $CC$ trop élevée est un code smell de Sonar Qube.
    \end{frame}

    \begin{frame}[fragile]
        \transdissolve
        \frametitle{Les mesures de la complexité cyclomatique}
        \framesubtitle{Exemple de fonction}
        % C code listing
        \begin{lstlisting}[language=c]
uint64_t fsma(uint64_t base, uint64_t exp, uint64_t mod) {
    uint64_t res = 1;
    while (exp > 1) {
        // If the exponent digit is 1, then multiply
        if (exp & 1) {
            res = (res * base) % mod;
            // If an intermediate result is zero, we can return 0.
            // The result will be zero anyway
            if (res == 0) {
                return 0;
            }
        }
        // If the exponent digit is 0, then square
        base = (base * base) % mod;
        exp >>= 1;
    }
    return (base * res) % mod;
}
        \end{lstlisting}

        Ici $E$ est 2 et $N$ est 3, donc $CC$ est 1.
    \end{frame}

    \begin{frame}[fragile]
        \transdissolve
        \frametitle{Les mesures de la complexité cyclomatique}
        \framesubtitle{Exercice}
        Calculer la complexité cyclomatique des fonctions de l'implémentation de l'algorithme de hachage MD5 du noyau Linux.
        \bigbreak
        Le code est sur Github~: \url{https://github.com/torvalds/linux/blob/master/crypto/md5.c}
        \bigbreak
        Rendre un tableau des valeurs de $CC$ pour chaque fonction.
    \end{frame}


    \section{Les différentes architectures d’une application}

\end{document}
